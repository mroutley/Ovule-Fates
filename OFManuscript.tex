\documentclass[letterpaper,titlepage]{scrartcl}
\usepackage{natbib,graphicx,amsmath,wasysym,amssymb}%,rotating}
\usepackage[utf8]{inputenc} 
% \usepackage[colorlinks=true,pdfstartview=FitV, linkcolor=blue,
% citecolor=blue, urlcolor=blue]{hyperref}

\setcounter{secnumdepth}{0} \raggedright \setlength{\parindent}{1em}
\pagestyle{plain} \renewcommand{\bibname}{Literature cited}
\bibliographystyle{cbe}

\newcommand{\species}[1]{\textit{#1}} \newcommand{\fref}{fig.}
\newcommand{\tref}{table} \newcommand{\eref}{equation}
\newcommand{\OFMixedMating}{L.~D.~Harder, S.~A.~Richards and
  M.~B.~Routley, unpublished manuscript}

\author{Matthew B. Routley\thanks{routley@ucalgary.ca} \and Lawrence
  Harder\thanks{harder@ucalgary.ca}} \title{Consequences of
  alternative ovule fates for seed production, maternal fitness, and
  mating-system evolution in flowering plants} \publishers{Department
  of Biological Sciences, University of Calgary, Calgary, Alberta,
  Canada T2N 1N4\\Short title: Ovule fates}
% \date{}

\begin{document}

\maketitle

\begin{abstract}
  Ovules are the progenitors of seeds and, consequently, are a crucial
  component of plant reproduction. We construct a model of the fates
  of ovules once they are produced and consider the consequences of
  these fates for seed production and fitness. Our model reveals three
  limits on seed production: pollen, resources, and ovules. Ovule
  limitation, in particular, is a novel limit on seed production that
  may have important consequences for the evolution of plant
  reproduction. In addition, our model clarifies important concepts in
  plant reproduction including ovule vs.~seed discounting, competing
  self-fertilization, pollen limitation, and reproductive
  assurance. We also illustrate conditions that favour mixed mating as
  an optimal strategy and suggest that simultaneous self-fertilization
  is more important than previously appreciated.
\end{abstract}

\section{Introduction}
As the progenitors of seeds, ovules link successive sporophyte
generations and so are a key stage in angiosperm life cycles. Ovule
production and fertilization govern the dynamics of plant populations
by establishing the maximum possible cohort size \citep{Moles04}. In
addition, the outcome of ovule fertilization determines the genetic
diversity of a (sporophyte) plant's offspring and so influences the
scope for selection on reproductive traits, especially the mating
system. Given the central role of ovules in angiosperm reproduction,
the variety and causes of ovule fates bear diverse and significant
ecological and evolutionary consequences.

Ovule fates are varied and interact in manners that enhance
reproductive options. Three main fates are immediately obvious. First,
some ovules may remain unfertilized because of limited pollen
dispersal \citetext{reviewed by \citealp{Burd95,Ashman04a}}. Although
fairly common, such pollen limitation constrains reproductive capacity
and is universally disadvantageous. The second major fate of ovules
arises as some embryos die after fertilization. These losses are often
significant, as an average of only 60\% of the ovules produced by
outcrossing species develop into seeds, even after abundant
pollination ({\fref}~\ref{fig:empirical}). Embryo deaths bear diverse
evolutionary implications, because they result from many processes,
including the expression of lethal genes, competition with other
embryos for maternal resources, and maternal offspring choice
\citep{Charlesworth87,Rigney95,Casper88}. The final major ovule fate is successful development into a
viable seed, which is of primary ecological and evolutionary
significance.

This trichotomy of fates is greatly complicated for self-fertile
hermaphroditic species, because self- and cross-fertilized embryos and
seeds differ in their value to the parental plant due to dissimilar
survival probabilities resulting from inbreeding depression
\citep{Charlesworth87} and the twofold greater contribution of a
specific parent's genes to the next generation through successful
self-fertilized offspring \citep{Fisher41}. In particular, as
\citet{Lloyd79,Lloyd92a} clarified, the consequences of
self-fertilization depend on when it occurs relative to
cross-fertilization and whether it reduces cross-fertilization (ovule
discounting) and the production of outcrossed seeds (seed
discounting). For example, \citet{Lloyd92a} proposed that
self-fertilization after cross-fertilization (delayed selfing) evolves
more readily than self-fertilization that precedes cross-fertilization
(prior selfing), because delayed selfing cannot cause ovule
discounting. Such differences in the consequences of the various modes
of self-fertilization expand the options for mating-system evolution.

Lloyd's \citeyearpar{Lloyd79,Lloyd92a} exploration of the modes of
selfing highlighted the relevance of ovule fates for mating-system
evolution; however, his analysis simplified or overlooked several
features of seed production that affect these fates. Lloyd recognized
five modes of self-pollination, but considered the consequences of
each mode in isolation, so that the interacting effects of ovule fates
remain largely unexamined. Furthermore, although \citet{Lloyd79}
recognized that “…the frequency of self-fertilization may greatly
exaggerate the role of self-fertilization in natural populations”
(pg. 76), because selfed embryos have lower survival than outcrossed
seeds, he summarized inbreeding depression in a single parameter. This
approach ignores environmental influences on inbreeding depression and
differences in the genetic causes and intensity of inbreeding
depression that affect selfed embryos and independent selfed offspring
\citep{Husband96}. Lloyd also evaluated the evolutionary consequences
of the modes of self-fertilization by contrasting the consequences of
no or complete pollen and seed discounting, rather than allowing their
intensities to be determined by mating circumstances. Finally, Lloyd
did not consider the possibility that many plants produce “excess”
ovules to compensate for embryo deaths during development, as
suggested by low seed production compared to ovule number, even when
pollen is abundant \citetext{{\fref}~\ref{fig:empirical};
  \citealp{Wiens84,Wiens89}}. If ``excess'' ovules compensate for
embryo deaths, then the loss of an embryo need not affect seed
production. Such reproductive compensation allows seed production to
be limited by ovule production, rather than by pollen receipt or the
resources available for seed development
({\OFMixedMating}). Furthermore, if compensatory ovules accommodate
the deaths of genetically inferior embryos without affecting seed
production, surviving selfed seeds can be more valuable than
outcrossed seeds, despite considerable genetic load in a population
\citetext{\citealp{Porcher05}; {\OFMixedMating}}. Clearly, the details
and consequences of ovule fates are more varied than revealed by
Lloyd's \citeyearpar{Lloyd79,Lloyd92a} seminal analyses.

In this paper, we model the diverse fates of ovules, including their
interacting effects on seed production and maternal fitness, and their
consequences for the evolution of plant mating. This approach provides
clarification of and insights into foundational concepts in plant
reproduction, including ovule vs.~seed discounting, competing
self-fertilization, reproductive assurance, pollen limitation, and
mixed mating.

\section{Ovule fates models}
We present two models of ovule fates that consider the proportions and
numbers of a plant's ovules involved in various outcomes,
respectively. Proportional fates are easier to understand (e.g.,
{\fref}~\ref{fig:fates}), so that we present this perspective in most
of this paper. However, proportional fates do not allow accurate
characterization of resource competition by developing embryos, which
depends on the numbers of embryos in different classes and the amount
of available resources. Therefore, we used the model that considers
the number of ovules subject to different fates, described in the
Appendix, for all calculations and examples. Conclusions drawn from
both models are equivalent. The variables and parameters considered in
these models are summarized in {\tref}~\ref{table:symbols}: note that
lower-case symbols represent proportions, whereas upper-case symbols
represent counts.

\subsection{Mode and timing of fertilization}
In general, the fates of the $O_{T}$ ovules produced by a plant depend
jointly on the outcomes of fertilization and seed development
({\fref}~\ref{fig:fates}). Self-fertilization can occur during three
qualitatively distinct phases: before, simultaneously with, or after
cross-fertilizations ({\tref}~\ref{table:modes}). Following
\citet{Lloyd92a}, we refer to self-fertilization before and after
cross-fertilization as prior and delayed self-fertilization,
respectively. Self-fertilization that occurs simultaneously with
cross-fertilization can result from one of three mechanisms of
self-pollination, two of which we identify differently from
\citeauthor{Lloyd92a} \citetext{\citeyear{Lloyd92a}: see
  {\tref}~\ref{table:modes}}. Autonomous simultaneous self-pollination
involves pollen that was deposited on the stigma without the action of
a vector. Alternatively, a pollen vector can facilitate
self-pollination, either within a flower (facilitated intrafloral
self-pollination) or between flowers on the same plant
(geitonogamy). Because the two modes of facilitated self-pollination
have similar consequences for ovule fates, we refer to the resulting
embryos jointly as the product of facilitated self-fertilization. In
addition to the different modes of self-fertilization, ovules can be
cross-fertilized or not fertilized by the end of flowering
({\fref}~\ref{fig:fates}).

The diversity of ovule fates depends on the fertilization sequence and
proportion of ovules that remain unfertilized at each stage in the
sequence ({\fref}~\ref{fig:fates}). Suppose that prior
self-fertilization removes a proportion, $p$, of a plant's
unfertilized ovules. If ovules remain unfertilized once flowers begin
interacting with pollen vectors (i.e., $p<1$), then proportions $a$,
$f$ and $x$ of the remaining $(1 - p)O_{T}$ ovules can be involved in
autonomous simultaneous self-fertilization, facilitated
self-fertilization (both intrafloral and geitonogamy), and
cross-fertilization, respectively. We assume that autonomous,
simultaneous self-fertilization always occurs independently of
cross-fertilization, as long as some ovules remain unfertilized. In
contrast, three relations between the incidence of facilitated self-
and cross-fertilization are possible, depending on the details of
pollination. First, facilitated self-fertilization could occur
independently of the delivery of cross-pollen (e.g., geitonogamy in a
wind-pollinated species), so that augmentation of facilitated
self-pollination increases $f+x$. Second, facilitated
self-fertilization may replace cross-fertilization (e.g., geitonogamy
by an animal that delivers a fixed pollen load), so that $x$ varies
negatively with $f$. Third, facilitated self-fertilization and
cross-fertilization could vary positively if self- and
cross-pollination both increase with each pollinator visit and plants
experience different pollinator visitation.  An investigation of these
three possible relations (results not shown) found that they had
little effect on the general conclusions drawn from the ovule fates
model. Consequently, although we recognize these three possible
relations, in practice we consider only autonomous simultaneous
self-fertilization to simplify the analysis. If any ovules remain
unfertilized once flowers stop interacting with pollen vectors (i.e.,
$p<1$ and $a+x<1$), then a fraction, $d$, of the remaining
$(1-p)(1-a-f-x)O_{T}$ ovules may be involved in delayed
self-fertilization, with the remaining fraction $(1-d)$ being
unfertilized. Given the preceding fates, a total of
\begin{equation}\label{eq:os}
  O_{S}=(p + [1 - p][a + d\{1 - a - x\}])O_{T}
\end{equation}
ovules have been self-fertilized, and
\begin{equation}\label{eq:ox}
  O_{X}=(1 - p)xO_{T}
\end{equation}
ovules have been cross-fertilized.

Seed development begins once ovules are fertilized. We propose that
some embryos fail to develop into viable seeds for three reasons:
genetic death, ``competition'' for resources, and predation
({\fref}~\ref{fig:fates}). Some self- and cross-fertilized embryos may
die shortly after fertilization without extracting appreciable
resources from the ovary, because of the expression of deleterious
alleles. Specifically, proportions $g_{s}$ and $g_{x}$ of the self-
and cross-fertilized embryos survive genetic death, with $g_{s}\leq
g_{x}$, because the greater homozygosity of self-fertilized embryos
increases the frequency of expression of recessive lethal traits
(pre-dispersal, or early acting inbreeding depression). In general,
the magnitude of $g_{s}$ and $g_{x}$ depend on the load of deleterious
alleles within the population \citep{Charlesworth87,Klekowski88}. Of
the remaining embryos, proportions $k_{s}$ and $k_{x}$ survive to
become seeds. These survival proportions depend on the proportion of
ovules that can mature into seeds given the resources available to the
maternal plant ($m$), the characteristics of developing embryos, and
the incidence of seed predation. Relevant embryo characteristics
include: whether the embryo was self-or cross-fertilized
\citep{Husband96}; when it was fertilized relative to other embryos
\citep{UmaShaanker95}; its intrinsic strength as a resource sink, perhaps
determined genetically \citep{Rigney95}; its relatedness to other embryos
\citetext{sibling rivalry and kin selection:
  \citealp{Kress81,UmaShaanker88}}; and maternal ``preference''
\citetext{offspring choice:
  \citealp{Casper88,Marshall88,Melser01}}. We will address the
consequences of the diverse competitive processes and seed predation
elsewhere. For the purposes of this paper, we consider only the
aggregate probabilities of successful development by self- and
cross-fertilized embryos ($k_{s}$ and $k_{x}$, respectively).

\subsection{Limits on seed production}
A plant's seed production depends on the abundance of three
components, pollen, ovules and resources, so that its female fertility
could be pollen-, ovule-, or resource-limited ({\fref}~\ref{fig:limits}A and
D). Pollen limitation occurs when fewer ovules are fertilized than
could develop into seeds given the available resources, even if no
embryos are lost to genetic death ($O_{s}+O_{x}<mO_{T}$:
{\fref}~\ref{fig:limits}A and D, lightly-shaded area). Ovule
limitation results when all ovules are fertilized, but too few embryos
survive genetic deaths to consume the available resources
($O_{s}g_{s}+O_{x}g_{x}<mO_{T}\leq O_{s}+O_{x}$:
{\fref}~\ref{fig:limits}A and D, thick solid line). Neither pollen nor
ovule limitation results in competition for maternal resources, so
that $k_{s}=k_{x}=1$ in the absence of maternal choice and seed
predation. Resource limitation results when
$O_{s}g_{s}+O_{x}g_{x}>mO_{T}$ and all surviving embryos compete for
maternal resources ({\fref}~\ref{fig:limits}A and D, heavily-shaded
area).

When maternal resources limit seed production, the probabilities that
self- and cross-fertilized embryos become seeds, depend on the number
of seeds that can be produced given the available resources
($mO_{T}$), the number of embryos that survive genetic death
($O_{s}g_{s}+O_{x}g_{x}$), and the competitive ability of
self-fertilized embryos relative to cross-fertilized embryos ($0\leq
\kappa \leq 1$). We depict competition among developing embryos as a
stochastic weighted lottery, whereby the ultimate proportion of
self-fertilized seeds depends on the proportion of surviving embryos
that were self-fertilized weighted by their relative competitive
ability. The noncentral hypergeometric distribution \citep{Chesson76}
describes the distribution of possible outcomes from such a
process. For this distribution, the expected (mean) number of
self-fertilized seeds ($E[S_{s}]$) given that $O_{s}g_{s}+O_{x}g_{x}$
embryos survive genetic death is calculated iteratively according to
\begin{equation*}
  q_{i}=q_{i-1}+\frac{(O_{s}g_{s}-q_{i-1})\kappa}{(O_{s}g_{s}-q_{i-1})\kappa +O_{x}g_{x}-(i-q_{i-1}-1)},
\end{equation*}
for $i=0$ to $mO_{T}$, where $q_{0}=0$. Overall, the expected
proportions of self- and cross-fertilized embryos that become seeds,
given that they survive genetic death are
\begin{equation*}
  k_{s}=
  \begin{cases}
    1 & \text{if } O_{s}g_{s}+O_{x}g_{x}\leq mO_{T}\\
    \frac{E[S_{s}]}{O_{s}g_{s}} & \text{if }
    O_{s}g_{s}+O_{x}g_{x}>mO_{T}
  \end{cases},
\end{equation*}
\begin{equation*}
  k_{x}=
  \begin{cases}
    1 & \text{if } O_{s}g_{s}+O_{x}g_{x}\leq mO_{T}\\
    \frac{mO_{T}-E[S_{s}]}{O_{x}g_{x}} &\text{if }
    O_{s}g_{s}+O_{x}g_{x}>mO_{T}
  \end{cases},
\end{equation*}
respectively. Note that if $\kappa =0$, this characterization of
resource competition describes the widespread phenomenon of cryptic
self-incompatibility \citep{Bertin89,Cruzan93,Rigney93,Eckert97},
because a plant produces self-fertilized seed following abundant
pollination with only self-pollen, whereas it produces only
cross-fertilized seed after abundant pollination with a mixture of
self- and cross-pollen.

\subsection{Pollen export}
Although we are primarily concerned with the fates of ovules and
maternal fitness in this paper, we cannot ignore the effects of
selfing on the pollen available for export \citetext{pollen
  discounting; \citealp{Harder98}; for further discussion of pollen
  fates, see \citealp{Harder00a}}. In this context, we assume that a
proportion ($\pi$) of pollen grains that could have been exported are
used in self-pollination. Consequently, the proportion $1-\pi$ grains
are exported to the pollen pool
% and a proportion $\epsilon$ are successfully deposited on
% stigmas. Therefore, the proportion of pollen grains that are
% successfully exported is given by the product $(1-\pi)\epsilon$.
We assume that the outcross siring success of a plant ($O_{\mars}$) is
determined by the plant's contribution to the total pool of exported
pollen.

% In this case, the paternal success of individual i in a population
% of $n$ plants is $\frac{(1-\pi_{i})}{\sum_{j=1}^{n}(1-\pi_{j})}$
% assuming that all plants have the same efficiency of pollen export
% ($\epsilon$), but differ in their proportions of discounting
% self-pollination ($\pi$).

\subsection{Fitness}
The fitness of a plant that produces $S_{s}=O_{s}g_{s}k_{s}$
self-fertilized seeds, $S_{x}=O_{x}g_{x}k_{x}$ cross-fertilized seeds,
and sires $S_{\mars}=O_{\mars}g_{x}k_{x}$ outcrossed seed, equals
\begin{equation}\label{eq:fitness}
  w=2S_{s}d_{s}+S_{x}d_{x}+S_{\mars}d_{x},
\end{equation}
where $d_{s}$ and $d_{x}$ are the probabilities that self- and
cross-fertilized seeds establish reproductive offspring,
respectively. {\eref}~\ref{eq:fitness} recognizes that self-fertilized
seeds carry two copies of maternal genes, whereas cross-fertilized
seeds bear only one copy \citep{Fisher41}. In general, $d_{s} \leq
d_{x}$, due to post-dispersal (late acting) inbreeding
depression.% Our analysis of mating-system evolution focuses on
% variation in maternal fitness, because we are interested in the
% specific consequences of different ovule fates. This approach
% assumes that all plants have equivalent paternal fitness.

\section{Results of the ovule fates model}

\subsection{Seed production and maternal fitness}
The outcome of any pattern of ovule fertilization can be quantified by
either fitness or seed production. Fitness governs the selection of
ovule fates and so provides an essential evolutionary
perspective. Seed production is a key demographic variable
\citetext{e.g., \citealp{Watkinson78,Bakker96}} and provides an
important ecological perspective. Furthermore, seed production is
measured easily and frequently. Therefore, we consider both
perspectives. To focus on the influence of ovule fates on maternal
fitness, in this section we assume that no pollen discounting occurs
($\pi =0$). We relax this condition later when we consider invasion
criteria for the evolution of the modes of selfing.

We illustrate the influence of changes in cross- and
self-fertilization for an ovary with $O_{T}=100$ ovules, of which
$mO_{T}=80$ can mature into seeds, given the available
resources. During flowering, this ovary receives at least 10 pollen
tubes each of prior self-, simultaneous self-, delayed self- and
cross-pollen (point 0 in {\fref}~\ref{fig:limits}A and D), unless
noted otherwise. From this baseline we increase the incidence of
either self-pollination (transect 0--3 in {\fref}~\ref{fig:limits}A
and D) or cross-pollination (transect 0--6 in
{\fref}~\ref{fig:limits}A and D) and assess the consequences for ovule
fates. Such increases in fertilization can cause complex changes in
seed production and fitness ({\fref}~\ref{fig:limits}B, C, E and F),
because early modes of self-fertilization usurp ovules that could have
been fertilized by later modes. We therefore focus on the effects of
changing the density of prior self-pollen, as it has the most diverse
effects on fertilization outcomes. As will be apparent, the relations
of seed production to increases in later-acting modes of
self-pollination are less complex.

We consider two cases, in which the proportions of self-fertilized
embryos surviving genetic death are either $g_{s}=0.3$ or 0.7. In both
cases, 90\% of cross-fertilized embryos survive genetic death
($g_{x}=0.9$) and self-fertilized embryos are 70\% as competitive as
cross-fertilized embryos when resources limit development ($\kappa
=0.7$). Finally, we assume that selfed seeds are at least half as
likely to establish adult plants as outcrossed seeds
($d_{s}/d_{x}>1/2$: specifically $d_{s}=0.7$, $d_{x}=0.9$), because
this condition allows for the greatest variety of optimal mating
systems ({\OFMixedMating}). This high survival of selfed seeds
relative to outcrossed seeds may occur commonly, because
post-dispersal inbreeding depression tends to be much weaker than that
acting during seed development in partially outcrossing species
\citep{Husband96}. As we explain below, the situation with $g_{s}=0.3$
({\fref}~\ref{fig:limits}A) favors production of a mixture of selfed
and outcrossed seeds is optimal, whereas that with $g_{s}=0.7$
({\fref}~\ref{fig:limits}D) favors exclusive selfing.

\subsubsection{Seed production}
When few ovules have been fertilized, increased receipt of any class
of compatible pollen alleviates pollen limitation
({\fref}~\ref{fig:limits}B and E). In this situation, each
self-fertilized embryo increases expected seed production by $g_{s}$
seeds ({\fref}~\ref{fig:limits}B and E, dashed lines from the origin
to point 1) and each cross-fertilization adds $g_{x}$ seeds
({\fref}~\ref{fig:limits}B and E, solid lines from the origin to point
4). With sufficient pollination, this relation of seed production to
fertilization eventually changes for two reasons. If more embryos can
survive genetic death than can develop given the available maternal
resources, then seed production ceases to increase when the number of
fertilizations exceeds the transition from pollen limitation to
resource limitation ($O_{s}g_{s}+O_{x}g_{x}>mO_{T}$: e.g.,
{\fref}~\ref{fig:limits}E, point 4). Alternatively, if a plant
produces so few ovules that not enough embryos survive genetic death
to compete for maternal resources, the steep relation of seed set to
increasing fertilization ends once all ovules have been fertilized
(points 1 and 4 in {\fref}~\ref{fig:limits}B, point 1 in
{\fref}~\ref{fig:limits}E).

The sign and magnitude of changes in seed set with each additional
pollen-tube that enters the ovary in excess of the number needed to
fertilize all ovules depends on the product of two factors: the
increment in the expected number of cross-fertilizations ($c$) and the
difference in the probability that cross- and self-fertilized embryos
survive genetic death ($g_{x}-g_{s}$). First consider the consequences
of increasing prior self-pollination ({\fref}~\ref{fig:limits}B and E,
dashed lines). Between points 1 and 2 in {\fref}~\ref{fig:limits}B and
E, increased prior self-pollination fertilizes ovules that would
otherwise have been fertilized by delayed self-pollination. Because
this usurpation does not affect cross-fertilization, $c=0$ and seed
production is unaffected. Between points 2 and 3, increased
self-pollination displaces both simultaneous self-fertilization and
cross-fertilization, which occur with equal frequency in this
example. Consequently, each additional prior self-fertilization has a
50\% chance of usurping a later cross-fertilization, so $c=-0.5$ and
seed production declines until prior self-pollen fertilizes all ovules
(point 3). Once prior self-pollen tubes fertilize all ovules any
additional prior self-pollination does not alter either the genetic
mixture of fertilizations (i.e., $c=0$), or seed set (right of point
3).

Now consider the consequences of increasing cross-pollination once a
stigma receives enough pollen to fertilize all ovules
({\fref}~\ref{fig:limits}B, solid line). Between points 4 and 5, each
cross-fertilization claims an ovule that would otherwise have waited
for delayed self-pollination, so $c=1$ and seed set increases by
$g_{x} - g_{s}$ for each cross-fertilization until no ovules remain for
delayed self-fertilization (point 5). Between points 5 and 6,
increased cross-pollination aggravates competition between
simultaneous self-pollen tubes and outcross pollen tubes for
fertilizations. In this case, the probability of an additional
outcross pollen tube fertilizing an ovule is
$c=1/(P_{A} + P_{X})$, so that seed set increases weakly at an
ever-diminishing rate toward an asymptote at which all ovules that
remain unfertilized after prior self-pollination are cross-fertilized
($O_{T} - O_{P}$). How closely this asymptote is approached depends on
the number of embryos that survive genetic death and can mature into
seeds ($mO_{T}$), given the available resources. In the example in
{\fref}~\ref{fig:limits}B (solid line) $mO_{T} < O_{T} - O_{P}$, so
resource limitation stops this increase before the asymptote is
reached (point 6).

The relations of seed production to cross- and self-pollination
described above apply generally, as long as self-fertilized embryos
survive genetic death less frequently than cross-fertilized embryos
($g_{s} < g_{x}$) and the proportion of ovules that could mature into
seeds lies between the proportions of self-fertilized and
cross-fertilized embryos that survive genetic death ($g_{s} < m < 
g_{x}$). Under these conditions, maximum production of
cross-fertilized seed is resource limited, whereas that of
self-fertilized seed is ovule limited, so that extensive
self-fertilization results in lower seed production than extensive
cross-fertilization. Consequently, seed set in this situation is
maximized by exclusive cross-fertilization, which limits the loss of
embryos to genetic death. This outcome would also arise if production
of both cross-fertilized and self-fertilized seeds was ovule limited,
as long as cross-fertilized seeds are more likely to survive genetic
death than self-fertilized seeds ($g_{s}<g_{x}<m$).

\subsubsection{Maternal fitness}
The consequences of ovule fates for fitness are driven by the
preceding effects on seed production, which are modified by the
probabilities of seedling establishment ($d_{s}$ and $d_{x}$) and the
twofold advantage of viable selfed offspring
({\fref}~\ref{fig:limits}C and F). As with seed set, under pollen
limitation ($O_{s} + O_{x} < mO_{T}$) fitness varies linearly with
pollen receipt until either seed set becomes resource limited
($O_{s}g_{s} + O_{x}g_{x} > mO_{T}$; {\fref}~\ref{fig:limits}F, point
4) or all ovules are fertilized ({\fref}~\ref{fig:limits}F, point 1;
{\fref}~\ref{fig:limits}C, point 1 and 4), although the slope of this
relation is now $2g_{s}d_{s}$ for the addition of self-pollen and
$g_{x}d_{x}$ for increased cross-pollination. The effects of further
increases in pollen receipt depend on the product of two factors: the
increment in the expected number of cross-fertilizations ($c$) and the
difference in the expected fitness of cross- and self-fertilized
embryos ($g_{x}k_{x}d_{x} - 2g_{s}k_{s}d_{s}$).

When individual selfed zygotes are less valuable to the maternal plant
than an outcrossed zygote ($2g_{s}d_{s}<g_{x}d_{x}$), the general
relation of fitness to increasing self- and cross-fertilization
essentially parallels that described above for seed production
(compare {\fref}~\ref{fig:limits}B and C). The one exception occurs
with increasing cross-fertilization if enough outcrossed embryos
survive genetic death that seed production becomes resource limited
(point 6). In this case, seed output is fixed by resource
availability, rather than genetic death of embryos, so that different
combinations of selfed and outcrossed seeds can maximize seed
production. However, these different combinations have unequal
maternal fitness, because a selfed seed that has survived genetic
death is more valuable than an outcrossed seed, as long as
$2d_{s}>d_{x}$. Given this condition and $2g_{s}d_{s}<g_{x}d_{x}$ and
$g_{s}<m< g_{x}$, maternal fitness is maximized if a plant has just
enough ovules cross-fertilized to maximize seed set and it
self-fertilizes the remainder ({\fref}~\ref{fig:limits}C, point 6;
also see {\OFMixedMating}). However, fitness changes very gradually
with changes in cross-fertilization near this transition
({\fref}~\ref{fig:limits}C), so that a slight deviation from the
optimal mixed mating system would have little fitness consequence.

The relations of seed production and maternal fitness differ more when
selfed zygotes are more valuable than outcrossed zygotes
($2g_{s}d_{s}>g_{x}d_{x}$: compare {\fref}~\ref{fig:limits}E and
F). In this case, $g_{x}k_{x}d_{x}-2g_{s}k_{s}d_{s}<0$, so that
increases in self- and cross-pollination cause opposite changes in
maternal fitness ({\fref}~\ref{fig:limits}F) to those described above
for seed production ({\fref}~\ref{fig:limits}E) when seed production
is not pollen limited. The advantage of replacing an outcrossed embryo
with a selfed embryo in this situation is evident in two aspects of
{\fref}~\ref{fig:limits}F. First, exclusive self-fertilization is
favored (dashed line above solid line), even though it does not
maximize seed production. Second, maternal fitness declines as
cross-fertilization increases beyond the number needed to invoke
resource competition (point 4). Specifically, increased
cross-pollination initially aggravates the resource competition
experienced by selfed embryos that survive genetic death until all
ovules are fertilized (between points 4 and 5), then usurps ovules
that could have been fertilized by delayed self-pollen (between points
5 and 6), and eventually competes with more beneficial simultaneous
self-pollen for fertilizations (right of point 6). Clearly, seed
production is a poor measure of evolutionary performance in this
situation.

\subsection{Mating system parameters}
Once seed development is complete, a plant's total production of self-
and cross-fertilized seeds equals $S_{s}=O_{s}g_{s}k_{s}$ and
$S_{x}=O_{x}g_{x}k_{x}$, respectively. As a result, the proportion of
seeds that are cross-fertilized is
\begin{equation*}
  t=
  \begin{cases}
    \frac{O_{x}g_{x}}{O_{s}g_{s}+O_{x}g_{x}} & \text{if } O_{s}g_{s}+O_{x}g_{x}\leq mO_{T}\\
    \frac{O_{x}g_{x}k_{x}}{O_{s}g_{s}k_{s}+O_{x}g_{x}k_{x}} &\text{if
    } O_{s}g_{s}+O_{x}g_{x}>mO_{T}
  \end{cases}
\end{equation*}

Differential genetic death ($g_{s}<g_{x}$) and/or resource competition
($k_{s}<k_{x}<1$) result in self-fertilized embryos having a lower
probability of becoming a seed than cross-fertilized
embryos. Inbreeding depression represents this proportional reduction
in fitness, $\delta=(w_{x}-w_{s})/w_{x}$, where $w_{x}$ and $w_{s}$
are the respective fitnesses of cross- and self-fertilized progeny
\citep{Charlesworth87}. According to our model, the probability that a
self-fertilized ovule matures into a seed is $g_{s}k_{s}$ and the
corresponding probability for a cross-fertilized ovule is
$g_{x}k_{x}$. Therefore, the inbreeding depression experienced during
seed maturation is
\begin{equation*}
  \delta=\frac{g_{x}k_{x}-g_{s}k_{s}}{g_{x}k_{x}}
\end{equation*}

When seed production is pollen or ovule limited
($O_{s}g_{s}+O_{x}g_{x} \leqslant mO_{T}$), inbreeding depression
depends only on the relative genetic losses for self- and
cross-fertilized ovules as $k_{s}=k_{x}=1$. If instead, resource
availability limits the production of both self- and cross-fertilized
seeds ($O_{s}g_{s}+O_{x}g_{x}>mO_{T}$), inbreeding depression also
depends on the relative competitive abilities of self- and
cross-fertilized ovules ($\kappa$).

\subsection{Mating-system evolution}
If plants do not differ in pollen export, fitness differences depend
on seed production, seedling establishment, and the two-fold
contribution of maternal genes by self-fertilized seeds compared to
cross-fertilized seeds,
\begin{equation*}
  w=2O_{s}g_{s}k_{s}d_{s} + O_{x}g_{x}k_{x}d_{x}.
\end{equation*}

Harder et al. ({\OFMixedMating}) considers the influence of the mating
system on maternal fitness. Here we summarize the major conclusions of
that analysis. First, maternal fitness is maximized when all ovules
are fertilized, so that $O_{s}^{*}+O_{x}^{*}=O_{T}$. Second,
$g_{s}/g_{x}=0.5$ is a threshold between whether maternal fitness is
maximized by producing as many selfed seeds as possible
($g_{s}/g_{x}>0.5$, $t^{*}=0$), or by producing some outcrossed seeds
($g_{s}/g_{x}<0.5$, $0<t^{*}\leqslant 1$). When $g_{s}/g_{x}>0.5$ the
twofold genetic contribution of selfed seeds exceeds the relative
chance of genetic death and so they are more valuable than outcrossed
seeds. In contrast, when $g_{s}/g_{x}<0.5$ selfed seeds have such a
high probability of genetic death and more genetic contributions can
be achieved through outcrossed embryos. Third, if some outcrossing is
favored ($g_{s}/g_{x}<0.5$), then $g_{x}=m$ is a threshold between
whether production of only outcrossed seeds ($t^{*}=1$,
$g_{x}\leqslant m$) or a mixture of selfed and outcrossed seeds
($0<t^{*}<1$, $g_{x}>m$) is optimal. In the latter case, the optimal
maternal outcrossing rate,
\begin{equation}\label{eq:mixedMating}
  t^{*}=\frac{g_{x}}{m}\frac{m-g_{s}}{g_{x}-g_{s}},
\end{equation}
increases as the maximum proportion of ovules that can mature into
seeds ($m$) approaches the proportion of outcrossed ovules that
survive genetic death ($g_{x}$). When mixed mating is optimal, the
numbers of selfed and outcrossed embryos that survive genetic losses
exactly equals the number of seeds that can be matured given the
available resources, so that there is no competition for those
resources. An increase in self-fertilization above the optimum would
increase genetic losses, so that some resources would not be used in
seed production (ovule limitation). In contrast, an increase in
cross-fertilization would increase competitive losses of selfed
embryos that have survived genetic losses and so are quite valuable,
given that they possess two maternal haploid genomes.

This characterization of the optimal mating system does not
incorporate the modes of
self-fertilization. {\fref}~\ref{fig:mixedMating} plots different
sources of self-pollen delivered at the optimal ratio with
cross-pollen calculated from {\eref}~\ref{eq:mixedMating} above for
$g_{s}=0.3$ and other parameters as in {\fref}~\ref{fig:limits}. Under
the conditions depicted, the optimal ratio of cross- to self-pollen is
0.75. {\fref}~\ref{fig:mixedMating}A plots the relation of the
seed-ovule ratio to the pollen-ovule ratio and
{\fref}~\ref{fig:mixedMating}B plots the relation of fitness, measured
as the number of genomes/ovule, and the pollen-ovule ratio. Each line
represents a different source of self-pollen. As examples, the
$\triangledown$ symbol plots delayed self-pollen and cross-pollen
deposited at the appropriate ratio and the $\circ$ symbol plots equal
allocations of the three modes of self-fertilization to the optimal
ratio. {\fref}~\ref{fig:mixedMating} shows that the modes of
self-fertilization have indistinguishable effects on seed set and
fitness under pollen limitation. The modes diverge once the
pollen-ovule ratio reaches one. {\fref}~\ref{fig:mixedMating}A
shows that increases in prior self-pollen beyond the pollen-limitation
threshold causes a decline in seed production as ovule limitation
intensifies to a limit of $g_{s}O_{T}$ seeds. For the other modes of
self-fertilization, seed production plateaus at $mO_{T}$ as
cross-pollen can either compete with simultaneous self-pollen or
preempt delayed self-pollen to avoid ovule limitation of seed
production. Eventually, however, prior self-pollen deposition is
sufficient to preempt cross-pollen with mixed pollen deposition. Seed
production then becomes ovule limited as the loss of self-fertilized
embryos from genetic deaths reduces seed production below
$mO$. {\fref}~\ref{fig:mixedMating}B shows that prior,
delayed, and mixed sources of self-pollen each cause a decline in
fitness beyond the pollen-limitation threshold. Prior self-pollen
preempts cross-pollen so that, although pollen is received at the
optimal ratio, the optimal ratio of fertilized embryos is not
produced. Consequently, fitness declines linearly with prior
self-pollen and cross-pollen beyond a pollen-ovule ratio of 1 and
ultimately yields a fitness value of $2g_{s}O_{T}$. In the case of
delayed self-pollen, once pollen limitation is alleviated cross-pollen
preempts delayed pollen. Given the parameters of
{\fref}~\ref{fig:mixedMating}, once this preemption occurs seed
production exceeds the resource threshold ($mO_{T}$) and delayed
embryos are lost to resource competition. This continues until all
ovules are cross-fertilized (with 100 cross-pollen and 40 delayed
self-pollen grains) which gives a fitness value of $g_{x}mO_{T}$. As
with delayed self-pollen deposition, mixed pollen deposition beyond
the pollen-limitation threshold invokes resource competition as
sufficient cross-fertilized embryos are produced to exceed
$mO_{T}$. However, in this case once cross-pollen has preempted
delayed self-pollen, prior self-pollen preempts and simultaneous
self-pollen competes with cross-pollen. Consequently, the production
of self-fertilized ovules increases and since many of these are lost
to genetic deaths, competition for resources lessens. Ultimately,
fitness recovers the maximum found at the threshold between pollen and
resource limitation and is then compromised by the transition from
resource to ovule limitation causing a new fitness decline that
eventually stops at $2g_{s}O_{T}$ once all ovules are fertilized by
prior self-pollen. {\fref}~\ref{fig:ternary} plots the optimal mating
system as a combination of cross-pollen and the three modes of
self-pollen. As indicated by the numerous points of the plots, maximum
maternal fitness can be realized through a variety of pollen
contributions.

\subsection{Invasion criteria and pollen discounting}
To explore the influence of ovule fates and pollen discounting on
mating system evolution, we consider the fitness differential between
two phenotypes (a mutant and the resident phenotypes) that differ in
their patterns of self-fertilization:
\begin{equation}\label{eq:differential}
  w_{2}-w_{1}=2(S_{s2}-S_{s1})d_{s}+(S_{x2}-S_{x1})d_{x}+(S_{\mars2}-S_{\mars1})d_{x},
\end{equation}
and determine the conditions that allow phenotype 2 (the mutant) to
invade a population of phenotype 1 (the resident; $w_{2}-w_{1}>0$). We
assume that genetic load is the same for each phenotype (i.e.,
$g_{s2}=g_{s1}$). Also, to simplify the analysis, we consider the case
where $\kappa =1$, so that
$k_{s}=k_{x}=\frac{mO}{O_{s}g_{s}+O_{x}g_{x}}$. For each mode of
selfing, we consider four invasion scenarios: no pollen discounting
and no resource competition; pollen discounting and no resource
competition; no pollen discounting and resource competition; and,
pollen discounting and resource competition.

\subsubsection{Non-discounting selfing}
Before considering specific modes of selfing, we analyze the general
conditions where selfing does not cause any pollen discounting
($\pi=0$). In these cases, $S_{\mars2}=S_{\mars1}$ and only maternal
fitness differs between the phenotypes. In the absence of resource
competition, $k_{s}=k_{x}=1$ and $w_{2}-w_{1}>0$ if,
\begin{equation}\label{eq:noDiscountNoResource}
  \frac{2g_{s}d_{s}}{g_{x}d_{x}}>\frac{O_{x1}-O_{x2}}{O_{s2}-O_{s1}}.
\end{equation}

The mutant can invade if the ratio of twice the probability of
selfed to crossed offspring survival exceeds the ratio of the mutant's
decrease in cross-fertilization to increase in self-fertilization. In
other words, if self-fertilization reduces cross-fertilization, selfed
offspring must be at least half as fit as crossed offspring. This is
equivalent to the classic inbreeding depression threshold of 1/2
\citep{Lloyd79,Lande85}. Furthermore, if increases in
self-fertilization increase the number of fertilized ovules without
compromising outcrossed embryo production, self-fertilization is
favoured. This is the reproductive assurance hypothesis for the
evolution of selfing.

When resource competition occurs, the values of $k$ change with the
number of ovules that are fertilized. Consequently we cannot assume
that the mutant and resident phenotypes experience equivalent
competition for resources. If $w_{2}-w_{1}>0$, then
\begin{equation*}
  \frac{2O_{s2}g_{s}d_{s}+O_{x2}g_{x}d_{x}}{O_{s2}g_{s}+O_{x2}g_{x}}mO - \frac{2O_{s1}g_{s}d_{s}+O_{x1}g_{x}d_{x}}{O_{s1}g_{s}+O_{x1}g_{x}}mO>0,
\end{equation*}
which simplifies to
\begin{equation}\label{eq:noDiscountResource}
  (O_{s2}O_{x1}-O_{s1}O_{x2})(2d_{s}-d_{x})>0.
\end{equation}

{\eref}~\ref{eq:noDiscountResource} is satisfied if
\begin{equation}\label{eq:noDiscountResourceCondition1}
  2d_{s}>d_{x} \text{ and } \frac{O_{s2}}{O_{x2}}>\frac{O_{s1}}{O_{x1}},
\end{equation}
or, if
\begin{equation}\label{eq:noDiscountResourceCondition2}
  2d_{s}<d_{x} \text{ and } \frac{O_{s2}}{O_{x2}}<\frac{O_{s1}}{O_{x1}}.
\end{equation}

The conditions of {\eref}~\ref{eq:noDiscountResourceCondition1} mean
that if the post-dispersal survival of selfed offspring is greater
than half that of crossed offspring, the mutant can invade if the
increase in selfing increases the ratio of selfed to crossed
fertilizations. If the post-dispersal condition of
{\eref}~\ref{eq:noDiscountResourceCondition1} is not satisfied, then
the mutant can invade given the second condition of
{\eref}~\ref{eq:noDiscountResourceCondition2}. However, this condition
states that self-fertilization must increase the frequency of
cross-fertilization. Given that this is not possible in our model,
only the conditions of {\eref}~\ref{eq:noDiscountResourceCondition1}
govern the ability of the non-discounting mutant to invade with
resource competition. Consequently, with resource competition the
threshold for the invasion of a selfing mutant is reduced from
$\frac{g_{s}d_{s}}{g_{x}d_{x}} > \frac{1}{2}$ to $\frac{d_{s}}{d_{x}}
> \frac{1}{2}$ (compare {\eref}~\eqref{eq:noDiscountResource} and {\eref}~\eqref{eq:noDiscountNoResource}

\subsubsection{Discounting selfing}
We now consider the general invasion conditions for mutants with
discounting self-fertilization. In the absence of resource
competition, the selection differential ($w_{2}-w_{1}$) is greater
than zero if
\begin{equation}\label{eq:discountNoResource}
  \frac{2g_{s}d_{s}}{g_{x}d_{x}}>\frac{(O_{x1}-\pi_{1})-(O_{x2}-\pi_{2})}{O_{s2}-O_{s1}}.
\end{equation}
The right side of {\eref}~\eqref{eq:discountNoResource} is greater than
the right side of {\eref}~\eqref{eq:noDiscountNoResource} whenever
$\pi_{2}>\pi_{1}$. Consequently, as expected, discounting makes it
more difficult for a selfing mutant to invade the resident population.

With resource competition, the invasion condition for discounting
selfing is complex. The condition simplifies to:
\begin{equation}\label{eq:discountResource}
  \frac{d_{s}}{d_{x}}>\frac{1}{2}+\frac{(\pi_{2}-\pi_{1})(g_{s}O_{s2}+g_{x}O_{x2})}{2g_{s}(O_{s2}O_{x1}-O_{s1}O_{x2})}.
\end{equation}
The second term of the right hand side of
{\eref}~\eqref{eq:discountResource} determines if discounting makes the
evolution of selfing easier with resource competition (compare with
{\eref}~\eqref{eq:noDiscountResource}). This term is negative only if
$\pi_{2}<\pi_{1}$ and
$\frac{O_{s2}}{O_{x2}}<\frac{O_{s1}}{O_{x1}}$. Since neither of these
conditions can be met, discounting also reduces the opportunities for
a selfing mutant to invade the resident population under resource
limitation.

\subsubsection{Delayed selfing}
Although delayed selfing is last in the sequence of selfing modes,
here we consider it first because of its relative simplicity. Since
delayed selfing cannot discount outcross pollen, $\pi_{1}=\pi_{2}=0$
and $S_{\mars2}=S_{\mars1}$. We consider a potential invader with a
fraction of delayed selfing $d_{2}=d_{1}+\Delta$ and substitute
{\eref}~\eqref{eq:os} and {\eref}~\eqref{eq:ox} into
{\eref}~\eqref{eq:noDiscountNoResource} to obtain $g_{s}d_{s}>0$. So, a
delayed selfing mutant can invade the resident population, in the
absence of resource competition, if selfed offspring have any
fitness. This is the classic reproductive assurance hypothesis for the
evolution of selfing \citep{Darwin76,Herlihy02}. However, with
resource competition, the conditions of
{\eref}~\eqref{eq:noDiscountResource} are only favoured if the
post-dispersal survival of selfed offspring is greater than twice that
of crossed offspring. Consequently, the general expectation that
delayed selfing is always favoured \citep{Lloyd79} is not true if
embryos compete for resources.

\subsubsection{Simultaneous selfing}
When considering the invasion of simultaneous selfing, we assume that
simultaneous selfing necessarily decreases cross fertilizations, so
that for the mutant $a_{2}=a_{1}+\Delta$ and $x_{2}=x_{1}-\Delta$. In
the absence of both discounting and resource competition, substituting
$a_{2}$ and $a_{1}$ into {\eref}~\eqref{eq:os}, $x_{2}$ and $x_{1}$ into
{\eref}~\eqref{eq:ox}, and solving {\eref}~\eqref{eq:noDiscountNoResource}
yields $\frac{g_{s}d_{s}}{g_{x}d_{x}}>\frac{1}{2}$; again the classic
inbreeding depression threshold. With resource competition,
{\eref}~\eqref{eq:noDiscountResource} requires that simultaneous selfing
increases the ratio of selfed to outcrossed seed production and that
$\frac{d_{s}}{d_{x}}>\frac{1}{2}$.

With pollen discounting, we have $a_{2}=a_{1}+\Delta$,
$x_{2}=x_{1}-\Delta$, and $\pi_{2}=\pi_{1}+\Delta$. Substituting these
values into {\eref}~\eqref{eq:os} and {\eref}~\eqref{eq:ox} and solving
{\eref}~\eqref{eq:discountNoResource} yields
\begin{equation*}
  \frac{2g_{s}d_{s}}{g_{x}d_{x}}>\frac{p}{1-p}.
\end{equation*}
As the proportion of ovules fertilized by prior self-pollen
increases, the conditions for the invasion of simultaneous selfing become
increasingly stringent. This is because increased prior selfing
reduces the number of unfertilized ovules and, consequently, reduces
the opportunity for the mutant to alleviate pollen limitation.

Adding resource competition and pollen discounting greatly complicates
the conditions that favour the invasion of simultaneous
selfing. Substituting {\eref}~\eqref{eq:os} and {\eref}~\eqref{eq:ox} into
{\eref}~\eqref{eq:discountResource} gives the condition
\begin{equation}\label{eq:simultaneousDiscountResource}
  \frac{d_{s}}{d_{x}}>\frac{1}{2}+\frac{1}{2}\cdot\frac{g_{s}[p+(1-p)(a_{1}+\Delta)]+g_{x}(1-p)(x_{1}+\Delta)}{g_{s}(1-p)[(1-p)(x_{1}+a_{1})+p]}.
\end{equation}
As described with {\eref}~\eqref{eq:discountResource},
{\eref}~\eqref{eq:simultaneousDiscountResource} generally shows that
discounting and resource competition make the invasion conditions more
stringent. However, the magnitude of this change depends on the
parameter values of {\eref}~\eqref{eq:simultaneousDiscountResource}.

% ??Is this second option useful??  The second option for simultaneous
% selfing is that it occurs independent of cross fertilizations, so
% that for the mutant $a_{2}=a_{1}+\Delta$, $x_{2}=x_{1}$, and
% $\pi_{2}=\pi_{1}+\Delta$. Without discounting or resource
% competition, the mutant can invade if
% \begin{equation}
%\frac{g_{s}d_{s}}{g_{x}d_{x}}(1-p)(1-d)>0
%\end{equation}
%These conditions are satisfied if either prior or delayed selfing are
% not complete, meaning that ovules remain unfertilized in the
% resident population. Once again, this is reproductive
% assurance. With resource competition, the conditions of
% {\eref}~\eqref{eq:noDiscountResource} require that $p<1$ and
% $0<\Delta<x$. These conditions are also met if ovules remain
% unfertilized in the resident population.

% Now, we consider the effect of pollen discounting. Without resource
% competition, substituting $a_{2}$, $a_{1}$, $\pi_{2}$, and $\pi_{1}$
% into {\eref}~\eqref{eq:discountNoResource} yields
% \begin{equation}
%\frac{2g_{s}d_{s}}{g_{x}d_{x}}>\frac{1}{1-p}
%\end{equation}
%The conditions for resource competition from
% {\eref}~\eqref{eq:discountResource} are
% \begin{equation}
%\frac{d_{s}}{d_{x}}>\frac{1}{2}+\frac{1}{2}\cdot\frac{g_{s}[p+(1-p)(a_{1}+\Delta)]+g_{x}(1-p)x_{1}}{g_{s}(p-1)^{2}x}
%\end{equation}

\subsubsection{Prior selfing}
In the case of prior selfing, the mutant self fertilizes with a
frequency $p_{2}=p_{1}+\Delta$. In the absence of both pollen
discounting and resource competition,
{\eref}~\eqref{eq:noDiscountNoResource} simplifies to
\begin{equation}\label{eq:priorNoDiscountNoResource}
  \frac{2g_{s}d_{s}}{g_{x}d_{x}}>\frac{x}{1-a}.
\end{equation}
This condition increases the threshold inbreeding depression for the
invasion of selfing above 1/2 when ovules remain unfertilized in the
resident population. This result was alluded to by \citet{Lloyd79} but
not developed. Note that the right hand side of this condition can
never exceed 1/2, so when inbreeding depression is less than 1/2
complete prior selfing is favoured. As with the previous analyses,
invoking resource competition requires that prior selfing increases
the ratio of selfed to outcrossed seed production and that
$\frac{d_{s}}{d_{x}}>\frac{1}{2}$.

If prior selfing causes pollen discounting, but not resource
competition, the condition for invasion of the mutant phenotype is
\begin{equation}\label{eq:priorDiscountNoResource}
  \frac{2g_{s}d_{s}}{g_{x}d_{x}}>\frac{1-x}{1-a}.
\end{equation}
A comparison of {\eref}~\eqref{eq:priorNoDiscountNoResource} and
{\eref}~\eqref{eq:priorDiscountNoResource} again shows that pollen
discounting reduces the fitness of the mutant.

With resource competition and pollen discounting, the condition
becomes
\begin{equation}\label{eq:priorDicountResource}
  \frac{d_{s}}{d_{x}}>\frac{1}{2}+\frac{1}{2}\cdot\frac{g_{s}[p_{1}+\Delta+a(1-p_{1}-\Delta)]+g_{x}(1-p_{1}-\Delta)}{g_{s}x}.
\end{equation}
As with {\eref}~\eqref{eq:simultaneousDiscountResource},
{\eref}~\eqref{eq:priorDicountResource} shows that invasion is less
likely with discounting and resource competition. However, no further,
general predictions are possible.

\section{Dynamic state variable model}
In the previous models, we assumed that self pollen arrives before,
simultaneous with, or after cross pollen and considered the
consequences of these schedules of pollen delivery. We now take a
different, but complimentary, approach by constructing a dynamic state
variable model \citep{Clark00} of pollen delivery and allow the plant
to `choose' the optimal pollination strategy. This dynamic model
considers a plant with 50 ovules that flowers for ten days and each
day a plant chooses to cross- or self-fertilize. If the plant chooses
to cross-fertilize, a pollinator will arrive with some given
probability and deposit a specific number of pollen grains. The
dynamic model also incorporates a facilitated selfing parameter, so
that some fraction of the deposited grains are self grains. If the
plant chooses to self-fertilize, a specific number of self grains are
deposited and cross-pollen is excluded for that time interval. For a
given set of parameter values, the dynamic model begins at the last
time interval and calculates the decision that, for a given number of
self- and cross-fertilized ovules, would yield the maximum fitness
value. These decisions than propagate backwards in time, to create a
matrix of optimal decisions for each time interval and each
combination of ovule fertilizations, assuming that optimal decisions
are made in each future interval. Once we construct a decision matrix
for a given set of parameters, we then place a plant at the origin
with no fertilized ovules and track the trajectory of mating outcomes
as it makes optimal decisions at each time step
({\fref}~\ref{fig:trajectory}). Whenever a plant chooses to cross
fertilize, we draw a random value from a Poisson distribution, with a
mean of the expected cross-pollen deposition, to add variation to the
mating outcomes. We implemented this model in R \citep{R} and the
source code is available by request to the authors.

\section{Results of the dynamic model}
Representative trajectories from the dynamic model are presented in
{\fref}~\ref{fig:trajectory}. In \ref{fig:trajectory}A, the proportion
of facilitated self-pollen is set to the optimal ratio of cross- to
self-fertilizations (see {\eref}~\eqref{eq:mixedMating}). Unlike prior
self-pollen, simultaneous self-pollen does not pre-empt cross
pollen. Furthermore, cross-pollen cannot pre-empt simultaneous
self-pollen as efficiently as it can pre-empt delayed
self-pollen. Consequently, with this set of parameter values, the
plant continuously chooses cross pollination and moves almost directly
to the optimal mating system. A comparison of \ref{fig:trajectory}B
and C shows the response of plants to a switch from high cross-pollen
and low self-pollen delivery to low cross-pollen and high self-pollen
delivery. In both cases, plants choose to cross-fertilize until the
last time step, when they switch to delayed
self-fertilization. Although these patterns mimic the predictions of
reproductive assurance, note that the plants are choosing to
self-fertilize to reach the optimal mating system. They are not simply
maximizing seed production with low quality selfed
offspring. Comparing panels B and D demonstrates the effect of poor
pollinator reliability on the mating trajectory. Although the amount
of pollen deposited per visit is the same in both panels, since
cross-pollen delivery is rare, plants rely on self-pollen to fertilize
ovules until the last time step. In the case of panel D, the plant is
relying on reproductive assurance offered by selfing to fertilize
ovules. Panel E depicts mating trajectories when cross-pollen is
delivered reliably and both cross- and self-pollen arrive in small
increments.%  ??This panel shows a variety of outcomes. In particular,
% since the optimal mating system is predominantly outcrossing, plants
% generally choose to cross-fertilize. However, if cross-pollen does not
% arrive with sufficient frequency, the plants will switch to selfing to
% compensate.??
Finally, panel F demonstrates what could be considered a
classic scenario for reproductive assurance. The plant receives very
rare cross-pollen and large amounts of self-pollen. In this case, the
plant waits for cross-pollen to arrive and then, generally,
self-fertilizes at the end. Panel D represents the potential for
reproductive assurance in plants that can self-fertilize in small
increments \citetext{e.g., the changes in herkogamy found in
  \species{Aquilegia canadensis}; \citealp{Griffin00}} and panel F for
plants that self-fertilize in large amounts \citetext{e.g., delayed
  selfing through corolla abscission in \species{Mimulus guttatus};
  \citealp{Dole90}}.

\section{Discussion}

\subsection{Limits on fitness}
Our model of ovule fates identifies three potential limits on maternal
fitness: resource limitation, pollen limitation, and ovule
limitation. \citet{Haig88} argued that seed production represents a
compromise between resource and pollen limitation such that maternal
fitness is, on average, limited by both simultaneously. Our
identification of ovule limitation of maternal fitness shows that this
argument is incomplete. In general, maternal fitness is maximized when
all ovules are fertilized ($O_{x}+O_{s}=1$) so that pollen limitation
is never an optimal strategy. Rather, as argued in Harder et al
({\OFMixedMating}), the appropriate trade-off is between ovule and
resource limitation. This proposal is supported by the empirical data
of {\fref}~\ref{fig:empirical}. If plants are chronically pollen
limited, supplemental addition of cross-pollen should enhance seed
set. However, the relationship between ovule and seed production is
statistically indistinguishable for open- and cross-pollinated
plants. The significantly smaller intercept for the self-pollinated plants
indicates that the loss of embryos to genetic deaths is a common
feature of plant reproduction. Consequently, empirical investigations
of ovule limitation may provide new insights into the evolutionary
ecology of plant reproduction.

Rather than focusing on pollen limitation \citep{Ashman04a}, we argue
that a plant's seed production and maternal fitness contribution can
be limited by the quantity and/or quality of pollen that it receives
\citep{Ramsey00}. In our model, this occurs when fewer than $mO_{T}$
ovules are fertilized and survive genetic losses, rather than simply
when there are some unfertilized ovules. Furthermore, ovule limitation
can reduce a plant's seed production below the maximum possible even
with sufficient pollen receipt. For example, suppose that all of a
plant's ovules are fertilized, with 70\% cross-fertilization and 30\%
self-fertilization, and that 90\% of cross-fertilized embryos develop
into seeds ($g_{x}=0.9$) compared to only 60\% for self-fertilized
embryos ($g_{s}=0.6$). In this case, total seed production equals
($0.7[0.9]+0.3[0.6])O_{T}=0.81O_{T}$. If instead, all of the embryos
had been cross fertilized, then seed production would have been
$0.9O_{T}$, an 11\% improvement, even though all ovules were
fertilized in both cases. However, note that the first situation
results in a 10\% higher maternal contribution to fitness (0.99 versus
0.9), because self-fertilized seeds contribute two maternal genomes
compared to one for cross-fertilized seeds (in the absence of
subsequent inbreeding depression after seed dispersal). Therefore,
whether pollen quality limits maternal reproductive output can depend
on perspective (seed production versus genome
contributions). Regardless of perspective, the overall occurrence of
pollen limitation is subject to diverse influences on seed production,
including; the amount of pollen received by stigmas, the composition
of that pollen, and the relative success of different classes of
embryos in surviving genetic and competitive losses.

These three limits on seed production are influenced by the diversity
of pollen received. The least diverse response is provided by pollen
limitation. As demonstrated in figs.~\ref{fig:limits} and
\ref{fig:mixedMating} any source of pollen can alleviate pollen
limitation. Furthermore, our dynamic model shows that a variety of
outcomes are possible ({\fref}~\ref{fig:trajectory}) if the goal is to
fertilize all of the available ovules without concern to the ratio of
cross- and self-fertilized embryos produced. Resource limitation is
invoked when embryos are fertilized in excess of the resources
available for their maturation. According to our model, competition
for resources results in competitive losses of fertilized embryos. If
$\kappa<1$, these losses occur most heavily among the valuable
self-fertilized embryos that survive genetic deaths. Regardless of the
value of $\kappa$, the loss of embryos waste both allocated resources
and captured pollen. Given that $g_{x}>g_{s}$, cross-fertilization is
more likely to invoke resource competition. This is illustrated in the
solid line of {\fref}~\ref{fig:limits}F and the equal mix of
self-pollen line of {\fref}~\ref{fig:mixedMating}B. Consequently, we
might expect prior self-pollination to be favoured in contexts that
favour reductions in resource competition. In contrast, since genetic
deaths are more likely among self-fertilized embryos, ovule limitation
is more common with predominant self-fertilization. Therefore,
predominant delayed self-pollination would alleviate ovule
limitation. In fact, the tension between these two limits provides the
prediction that mixed mating is an optimum mating strategy in some
circumstances.

\subsection{Evolution of the mating system}
Our model of ovule fates provides several important insights into
mating system evolution in flowering plants. These insights include a
clarification of the terms seed and ovule discounting, the evolution
of mixed mating, and the importance of reproductive assurance.
Self-pollination can often reduce (discount) opportunities for
producing cross-fertilized seeds. However, there is a distinction
between ovule and seed discounting. Ovule discounting occurs when
self-pollination reduces the number of ovules that could have been
cross-fertilized \citep{Barrett02}. This process does not require
self-fertilization, because late-acting self-incompatibility systems
can disable ovules that interact with self-pollen tubes, even though
fertilization does not occur, precluding subsequent
cross-fertilization \citetext{reviewed by \citealp{Barrett02}}. Seed
discounting occurs when self-fertilization reduces a plant's seed
production \citep{Lloyd92a,Herlihy02}. Ovule discounting leads to seed
discounting only when seed production is pollen limited; otherwise,
self-fertilization does not reduce the production of cross-fertilized
seeds. Therefore, seed discounting requires ovule discounting, but
ovule discounting need not result in seed discounting. Because of this
asymmetry, it is valuable to distinguish between ovule and seed
discounting.

Our ovule fates model predicts that mixed mating can be an optimal
reproductive strategy, rather than an inevitable by-product of poor
pollinator service (reproductive assurance) or large floral displays
(geitonogamy). The details of this prediction are described in Harder
et al ({\OFMixedMating}). In the context of this paper, we explored
the influence of the type of pollen deposition on the optimal mixed
mating strategy ({\fref}~\ref{fig:mixedMating}). In this context, the
optimal mixed mating-system is realized by a proportion ($m-g_{s})/(
g_{x}-g_{s}$) cross-fertilized ovules and the remainder
self-fertilized leaving no unfertilized ovules. As illustrated in
{\fref}~\ref{fig:mixedMating} and {\fref}~\ref{fig:ternary}A this can
be attained by any mode of self-pollen if pollen deposition sums
precisely to a pollen-ovule ratio of one. Given the vagaries of pollen
deposition, this precision seems unlikely. Pollen deposition less than
a pollen-ovule ratio of 1 results in a sharp decline in fitness. With
the exception of simultaneous self-pollen, pollen deposition in excess
of a pollen-ovule ratio of one also causes a decline in fitness. This
decline is especially rapid for a mixture of pollen sources or delayed
self-pollen because an increase in cross-fertilized ovules invokes
resource competition. The prediction that simultaneous
self-fertilization is favored by natural selection for maintaining
mixed mating is surprising. This mode of self-fertilization is
comprised of autonomous simultaneous, facilitated intrafloral, and
geitonogamous sources of self-pollen. The latter two in particular are
usually considered to be inevitable by-products of adaptations for
pollinator attraction and the import and export of cross-pollen
\citep{Lloyd92}. Despite considerable attention received by
geitonogamy, few studies have measured the magnitude of simultaneous
intrafloral self-pollen. However, this mode of selfing links cross-
and self-pollen deposition which can be quite beneficial if mixed
mating is favoured. Floral traits, such as herkogamy and dichogamy,
that modify the availability of self-pollen may evolve to realize the
optimal ratio of cross- and self-pollen, rather than reduce the
frequency of inbreeding. This argument is supported by the observation
that these traits often do not eliminate self-fertilization
\citetext{e.g., \citealp{Griffin00}}.

Reproductive assurance is the dominant explanation for the maintenance
of self-fertilization when $g_{x}<2g_{s}$
\citep{Darwin76,Lloyd92a,Herlihy02,Kalisz04}. When pollinators are rare,
self-fertilization allows for the fertilization of ovules that would
remain unfertilized otherwise. In particular, delayed
self-fertilization is thought to be effective at providing
reproductive assurance \citep{Lloyd92a,Kalisz99}. However, as is
illustrated in {\fref}~\ref{fig:limits}C and F, under pollen
limitation all modes of self-fertilization are equally effective at
providing reproductive assurance. The benefit of delayed selfing is
that it cannot preempt cross-fertilization and cause ovule
discounting. However, we have identified two fallacies with the
invocation of delayed selfing as a mechanism for reproductive
assurance. The first is that delayed selfing is not universally
favoured. If the addition of more fertilized embryos invokes resource
competition, delayed selfing can be disadvantageous. Once resource
limitation is invoked, increased selfing is only favoured if
$2d_{s}>d_{x}$. The second fallacy is that the occurence of delayed
selfing is evidence for reproductive assurance. As demonstrated in
{\fref}~\ref{fig:trajectory}, delayed selfing can also be favoured for
realizing the optimal mixed-mating system. In this case, the plant is
manipulating the ratio of cross- to self-fertilizations, rather than
simply maximizing the number of fertilized ovules. These fallacies
highlight the care that must be taken when evaluating reproductive
assurance hypotheses for the evolution of self-fertilization.

\subsection{Empirical considerations}
Our model reinforces the fact that seed production does not always
equal fitness (compare {\fref}~\ref{fig:limits}B and D, E and F). In
general, seed production is maximized by cross-fertilization and
fitness (when $g_{x}<2g_{s}$) by self-fertilization. The two-fold
advantage of self-fertilization has been appreciated since
\citet{Fisher30}. Nonetheless, seed production is the predominant
measure used to quantify maternal fitness. Since many plant species
exhibit mixed mating \citep{Barrett90a} the realized fitness of
phenotypes could be underestimated. Ideally, measures of seed
production should be associated with estimates of the mating system so
that fitness can be accurately estimated.

The influence of pollen quality on a plant's seed production bears
important practical implications for the detection of pollen
limitation. The typical approach to assessing pollen limitation
usually involves supplementation of a flower's natural pollen receipt
with cross-pollen \citetext{reviewed by \citealp{Burd94}}. However, if
simultaneous or delayed self-fertilization occur commonly, then
pollen-supplemented flowers will receive a different mixture of pollen
than naturally pollinated flowers. This difference could result in
higher seed production by supplemented flowers, even though all ovules
in naturally pollinated flowers are fertilized, if self-fertilized
embryos have a lower chance of developing into seeds than
cross-fertilized ovules. Consequently, pollen supplementation may
often lead to a conclusion of limitation, even though none exists
\citetext{also see \citealp{Ramsey95,Ramsey00}}.

\section{Appendix: Correspondence between numerical and proportional ovule fates}
Ovules are discrete entities, which are often produced in relatively
small number. Therefore we use a numerical approach to calculating the
integer number of self- and cross-fertilized embryos, $F_{s}$ and
$F_{x}$, respectively, given that a plant produces a total of $O_{T}$
ovules. These ovules are fertilized by pollen tubes that arrive at
different times and represent different paternal plants, including
cross-pollen $P_{x}$, prior self-pollen $P_{p}$, autonomous
simultaneous self-pollen $P_{a}$, facilitated intrafloral self-pollen
$P_{f}$ (treated as a proportion of $P_{x}$; $0\leqslant
P_{f}\leqslant 1$), and delayed self-pollen $P_{d}$. These five
classes of pollen fertilize available ovules to produce prior ovules
$O_{p}$, autonomous simultaneous ovules $O_{a}$, facilitated
intrafloral ovules $O_{f}$, outcrossed ovules $O_{x}$, and delayed
ovules $O_{d}$ as follows:

\begin{equation}
  O_{p}=
  \begin{cases}
    P_{p},      &\text{if }P_{p} < O_{T};\\
    O_{T},      &\text{if }P_{p} \geqslant O_{T};\\
  \end{cases}
\end{equation}

\begin{equation}
  O_{a}=
  \begin{cases}
    P_{a}, & \text{if }P_{a} + P_{x} < O_{T} - O_{p};\\
    \frac{P_{a}}{P_{a} + P_{x}} \left( O_{T} - O_{p}\right), &\text{if }P_{a} + P_{x} \geqslant O_{T} - O_{p};\\
  \end{cases}
\end{equation}

\begin{equation}
        O_{f}=
        \begin{cases}
  P_{f}P_{x}, &\text{if }P_{a} + P_{x} < O_{T} - O_{p};\\
  \frac{P_{f}P_{x}}{P_{a} + P_{x}} \left( O_{T} - O_{p}\right), &\text{if }P_{a} + P_{x} \geqslant O_{T} - O_{p};\\
        \end{cases}
\end{equation}

\begin{equation}
  O_{x}=
  \begin{cases}
    P_{x}, &\text{if }P_{a} + P_{x} < O_{T} - O_{p};\\
    \frac{P_{x}}{P_{a} + P_{x}} \left( O_{T} - O_{p}\right), &\text{if }P_{a} + P_{x} \geqslant O_{T} - O_{p};\\
  \end{cases}
\end{equation}

\begin{equation}
  O_{d}=
  \begin{cases}
    P_{d}, &\text{if }O_{T} > O_{p} + O_{a} + O_{x}\\
    &\text{and}\\
    &\text{if }P_{d} < O_{T}-O_{p}-O_{x}-O_{a};\\
    O_{T}-O_{p}-O_{x}-O_{a}, &\text{if }O_{T} > O_{p} + O_{a} + O_{x}\\
    &\text{and}\\
    &\text{if }P_{d} \geqslant O_{T}-O_{p}-O_{x}-O_{a};\\
    0, &\text{if }O_{T}=O_{p}-O_{x}-O_{a};\\
  \end{cases}
\end{equation}

Note that autonomous simultaneous self-pollen and cross-pollen
fertilize ovules in proportion to their relative abundance. Given this
numerical assignment of ovules, the numbers of self-fertilized embryos
are:

\begin{equation}
  O_{s}=O_{p}+O_{a}+O_{d}
\end{equation}

The preceding results also allow us to calculate the conditional
probabilities of ovules being fertilized by prior selfing ($p$),
simultaneous autonomous selfing ($a$), delayed selfing ($d$) and
outcrossing ($x$):

\begin{equation}
  p=O_{p}/O_{T}
\end{equation}

\begin{equation}
  a=\frac{O_{a}}{O_{T}-O_{p}}
\end{equation}

\begin{equation}
  d =\frac{O_{d}}{O_{T}-O_{p}-O_{x}-O_{a}}
\end{equation}

\begin{equation}
  f =\frac{O_{f}}{O_{T}-O_{p}}
\end{equation}

\begin{equation}
  x =\frac{O_{x}}{O_{T}-O_{p}}
\end{equation}

\bibliography{/Users/mroutley/Documents/References}

\newpage
\section{Figure captions} {\fref}~\ref{fig:empirical}: Ovule and seed
production data are presented for 332 species. The figure plots ovules
per flower and seeds per flower on log scales. Results from open
pollination are plotted as solid circles, hand-cross pollination as
open squares, and hand-self pollination as open diamonds. The solid
line indicates a seed-ovule ratio of 1 and the dashed line a
seed-ovule ratio of 0.5. The solid grey line is the line of best fit
through the open-pollinated data (log(Seeds/flower) =
log(Ovules/flower)-0.27) and yields an average seed-ovule ratio of
0.534. The dashed grey line is the line of best fit through the
cross-pollinated data and the dot-dash grey line is for the
self-pollinated data.

{\fref}~\ref{fig:fates}: Decomposition of ovule fates. See
{\tref}~\ref{table:symbols} for definitions of symbols.

{\fref}~\ref{fig:limits}: General relations of pollen, ovule and
resource limitation of seed production to the proportions of self-
($O_{s}/O_{T}$) and cross-fertilized ovules ($O_{x}/O_{T}$: A and D)
and the relations of seed production (B and E) and maternal fitness (C
and F) to the relative abundance of pollen tubes in the ovary. In all
panels, 90\% of cross-fertilized embryos survive genetic death
($g_{x}=0.9$), a plant with $O_{T}=100$ ovules has sufficient
resources to mature 80\% of its ovules into seeds (m = 0.8), selfed
embryos are less competitive than outcrossed embryos ($\kappa =0.7$),
and 70\% of selfed and 90\% of outcrossed seeds become reproductive
adults ($d_{s}=0.7$, $d_{s}=0.9$). For A-C, 30\% of self-fertilized
embryos survive genetic death ($g_{s}=0.3$) and mixed-mating is
favored, whereas for D-F, 70\% survive and exclusively selfing is
favored. In A and D the solid diagonal line represents the
ovule-production constraint ($O_{s}+O_{x}=O_{T}$) and the dashed line
indicated the transition from pollen to resource
limitation($O_{s}g_{s}+O_{x}g_{x}=m$). The lightly shaded area
illustrates combinations of self- and cross-fertilization that result
in pollen limitation, whereas ovule limitation occurs on the solid
dark line along the ovule-constraint line. The darkly shaded area
illustrates combinations of self- and cross-fertilization that result
in resource limitation. In B, C, E and F, solid lines illustrate the
effects of changes in the number of cross-pollen tubes when there are
also 10 pollen tubes each produced by prior, simultaneous, and delayed
self-pollen. The dashed lines depict corresponding responses to
changes in the number of prior self-pollen tubes, when there are 10
pollen tubes of each of the other classes. Maternal fitness (C and F)
is measured as the number of genomes per ovule contributed to the next
generation. Numbers in all panels indicate transitions in either the
interactions of ovule fates or resource competition.

{\fref}~\ref{fig:mixedMating}: The relationships between seed
production and fitness for pollen deposition when pollen arrives in
the optimal ratio of cross and self-pollen. Each line differs in the
source of self-pollen, as indicated in the legend. In all cases
$g_{x}=0.9$, $g_{s}=0.3$, $m=0.6$, and $\kappa =0.7$.

{\fref}~\ref{fig:ternary}: Ternary plots indicating the combinations
of pollen that produce the maximum maternal fitness. The axes
represent the proportion of ovules fertilized by prior (P) and
autonomous (A) self pollen and cross-pollen (X). To simplify the
presentation, we only plot combinations of fertilizations that do not
include delayed self-fertilization. For example, at the lower right
vertex prior self- and cross-fertilized ovules are equally abundant
with no autonomously fertilized ovules. Panel A shows optimal
combinations for $g_{x}=0.9$, $g_{s}=0.3$, $m=0.6$, and $\kappa
=0.7$. Under these conditions, mixed mating is the optimal strategy
({\eref}~\eqref{eq:mixedMating}). Panel B shows optimal combinations for
$g_{x}=0.9$, $g_{s}=0.7$, $m=0.6$, and $\kappa =0.7$. Under these
conditions, pure self-fertilization is the optimal strategy.

{\fref}~\ref{fig:trajectory}: Trajectory plots of mating outcomes from
the dynamic model. Four representative trajectories are shown in each
panel. Each point of a trajectory represents a time interval and
plants make optimal decisions at each of ten time intervals given the
number of cross- and self-fertilized ovules. In all panels,
$g_{x}=0.9$, $g_{s}=0.3$, $d_{x}=0.9$, $d_{s}=0.7$, $m=0.8$, and
$\kappa =0.7$. The open square in each panel indicates the optimal
mating system given these parameters ($t=0.83$) and the dashed
diagonal line is the ovule production constraint. The proportion of
facilitated self-pollen is 0.17 in panel A and 0 in all others. In A
mean cross-pollen deposition is 5 grains, self-pollen deposition is 20
grains, and the probability of a pollinator visit in each visit is
0.5. These parameters for the other panels are as follows: B 20
cross-pollen grains, 5 self-pollen grains, and the probability of a
visit is 0.5; C 5 cross-pollen grains, 20 self-pollen grains, and the
probability of a visit is 0.5; D 20 cross-pollen grains, 5 self-pollen
grains, and the probability of a visit is 0.1; E 5 cross-pollen
grains, 5 self-pollen grains, and the probability of a visit is 0.9;
and F 1 cross-pollen grains, 20 self-pollen grains, and the
probability of a visit is 0.1. In panel F, the majority of
trajectories involve pure selfing. A small fraction ($\sim$ 10\%)
include cross-fertilization in the final time step.

\newpage


\begin{table}[p]
\begin{minipage}{\textwidth}
  \begin{tabular}{lcc}
    \hline
    Definition & Proportion & Number\\
    \hline
    Prior self-pollen &  & $P_{P}$\\
    Autonomous self-pollen &  & $P_{A}$\\
    Facilitated self-pollen &  & $P_{F}$\\
    Cross-pollen &  & $P_{X}$\\
    Delayed self-pollen &  & $P_{D}$\\
    Increment in cross-fertilization from an additional pollen tube & $c$ & \\
    Total ovules &  & $O_{T}=100$\\
    Cross-fertilized ovules & $x$\footnote{This parameter represents the proportion of ovules that remain unfertilized after prior self-fertilization, rather than the proportion of all ovules.} & $O_{X}$\\
    Prior self-fertilized ovules & $p$ & $O_{P}$\\
    Autonomous self-fertilized ovules & $a^{a}$ & $O_{A}$\\
    Facilitated self-fertilized ovules & $f^{a}$ & $O_{F}$\\
    Delayed self-fertilized ovules & $d$\footnote{This parameter represents the proportion of ovules that remain unfertilized after prior and simultaneous self-fertilization plus cross-fertilization, rather than the proportion of all ovules.} & $O_{D}$\\
    All self-fertilized ovules & & $O_{S}$\\
    Self-fertilized embryos surviving genetic deaths & $g_{s}=0.3$ or $0.7$ & \\
    Cross-fertilized embryos surviving genetic deaths & $g_{x}=0.9$ & \\
    Ovules that can mature into seeds & $m=0.8$ & \\
    Cross-fertilized embryos surviving competitive deaths & $k_{s}$ & \\
    Cross-fertilized embryos surviving competitive deaths & $k_{x}$ & \\
    Competitive ability of self-vs.~cross-fertilized embryos & $\kappa =0.7$ & \\
    Self-fertilized seeds &  & $S_{s}$\\
    Cross-fertilized seeds &  & $S_{x}$\\
    Self-fertilized seeds that become adults & $d_{s}=0.7$ & \\
    Cross-fertilized seeds that become adults & $d_{x}=0.9$ & \\
    Maternal fitness &  & $W$\\
    \hline
  \end{tabular}
  \caption{Summary of parameters and variables considered by the ovule fates models (see text and Appendix).  Numbers associated with parameters indicate default values considered in the numerical analysis.}
  \label{table:symbols}
\end{minipage}
\end{table}

\begin{table}[p]
\begin{minipage}{\textwidth}
  \begin{tabular}{p{2.3cm}p{3cm}ccp{3cm}}
    \hline
    Mode\footnote{We refer to ``autonomous, simultaneous
      self-fertilization'' and ``facilitated, intrafloral
      self-fertilization'' rather than Lloyd's (1992) terms, competing and
      facilitated self-fertilization, respectively, because they describe
      the relevant processes more accurately. Lloyd's reference to
      competing self-fertilization is misleading for two reasons. First,
      self- and cross-pollination and the resulting fertilizations are
      competitive only if more pollen tubes enter the ovary than are
      necessary to fertilize all available ovules and more ovules are
      fertilized than can be matured into seeds, given the available
      maternal resources. Second, when maternal resources are limited,
      ovules fertilized by any class of self-pollen, except perhaps
      delayed self-pollen, can compete with cross-fertilized embryos for
      resources during development. Lloyd's use of facilitated
      self-fertilization in reference only to an intrafloral process also
      introduces confusion, because geitonogamy also requires the action
      of a pollen vector. For these reasons, we prefer the more precise,
      if more cumbersome, terms for autonomous simultaneous and
      facilitated intrafloral self-fertilization.} & Timing relative to cross-pollen & Scale of pollination & Pollination mode & Lloyd's (1992) terminology\\
    \hline
    Prior & Before & Within flower & Autonomous & Prior\\
    Autonomous simultaneous & Simultaneous & Within flower & Autonomous & Competing\\
    Facilitated intrafloral & Simultaneous & Within flower & Facilitated & Facilitated\\
    Geitonogamy  & Simultaneous & Between flowers & Facilitated & Geitonogamy\\
    Delayed & After & Within flower & Autonomous & Delayed\\
    \hline
  \end{tabular}
  \caption{Functional definitions of the modes of self-fertilization and their correspondence with Lloyd's (1992) terminology.}
  \label{table:modes}
\end{minipage}
\end{table}

% \begin{table}[p]
%\begin{tabular}{r|ll}
%\hline
% & no discounting & discounting\\
%\hline
%no resource competition & $\frac{2g_{s}d_{s}}{g_{x}d_{x}}>\frac{O_{x1}-O_{x2}}{O_{s2}-O_{s1}}$ & $\frac{2g_{s}d_{s}}{g_{x}d_{x}}>\frac{O_{x1}-\pi_{1}-(O_{x2}-\pi_{2})}{O_{s2}-O_{s1}}$\\
%resource competition & $\frac{d_{s}}{d_{x}}>\frac{1}{2}$ & $\frac{d_{s}}{d_{x}}>\frac{1}{2}+\frac{(\pi_{2}-\pi_{1})(g_{s}O_{s2}+g_{x}O_{x2})}{2g_{s}(O_{s2}O_{x1}-O_{s1}O_{x2})}$\\
%\hline
%\end{tabular} 
%\caption{The invasion conditions for discounting and non-discounting selfing with and without resource competition.}
%\label{table:invasionConditions}
%\end{table}

\newpage

\begin{figure}[p]
  \includegraphics[scale=1]{OFEmpirical}
  \caption{}
  \label{fig:empirical}
\end{figure}

\begin{figure}[p]
  \includegraphics[scale=1]{OFFlow}
  \caption{}
  \label{fig:fates}
\end{figure}

\begin{figure}[p]
  \includegraphics[scale=1]{OFLimits}
  \caption{}
  \label{fig:limits}
\end{figure}

\begin{figure}[p]
  \includegraphics[scale=1]{OFMixedMating}
  \caption{}
  \label{fig:mixedMating}
\end{figure}

\begin{figure}[p]
  \includegraphics[scale=1]{OFTernary}
  \caption{}
  \label{fig:ternary}
\end{figure}

\begin{figure}[p]
  \includegraphics[scale=1]{OFTrajectory}
  \caption{}
  \label{fig:trajectory}
\end{figure}

\end{document}